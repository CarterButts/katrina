% ==============
% Begin preamble
% ==============
\documentclass[11pt]{article}
%\VignetteIndexEntry{Codebook for the katrina R Data Package} 
%\VignettePackage{katrina}

\usepackage[dvips]{graphics}
\usepackage{rotating}
\usepackage{datetime}
\usepackage{booktabs}
\usepackage{multirow}
\usepackage{amsmath}
\usepackage{multirow}
\usepackage{color}
\usepackage{verbatim}
\usepackage{cancel}
\usepackage{natbib}
\usepackage{float}

\oddsidemargin -.25in	% Lt Margin adj for odd pages (0 = 1" margin)
\evensidemargin -.25in	% Rt Margin adjustment for even pages
\topmargin -.5in	% adj top margin frhamom 1"
%\footskip .5in
\textwidth 7in	% width of printed area
\textheight 9in	% height of printed area

\pagestyle{plain}

%---Bibliography information
\usepackage{natbib}
\bibliographystyle{apalike}
\renewcommand{\bibsection}{\section{References}}
\bibpunct{(}{)}{;}{a}{,}{;}
% ============
% End preamble
% ============

\usepackage{Sweave}
\begin{document}

\title{\texttt{katrina} R Data Package Codebook}
\author{Carter T. Butts\footnote{Department of Sociology and Institute for Mathematical Behavioral Sciences; University of California, Irvine.} \and Ryan M. Acton\footnote{Department of Sociology; University of Massachusetts, Amherst.} \and Christopher Steven Marcum\footnote{Department of Sociology; University of California, Irvine.}}
\date{11/8/2010}
\maketitle

%---Actual Text Begins Here
\section{Overview}

\subsection{Philosophy}
The extraction of relational information from field documents is intrinsically difficult. Given the highly variable nature of the source documents, great care must be exercised to permit reliable coding within a reasonable time frame. To the extent possible, the following principles have been followed in pursing this end:
\begin{enumerate}
\item Reduce the use of human judgment wherever possible;
\item Limit the number of coding decisions which must be made during each stage of the coding process (i.e., code only one thing at a time, separate tagging and coding tasks);
\item Employ multiple coders for all tagging and coding tasks;
\item Assess inter-coder reliability for all tagging and coding tasks;
\item Independently verify all critical procedures;
\item Track all changes by both personnel ID and date of change;
\item Store data at each phase of the tagging/coding process;
\item Document all procedures.
\end{enumerate}
The end result of this process, it is hoped, is a data set of much greater quality and scope than could be obtained using ad hoc methods.

\section{Raw Material Collection}
Raw materials were collected from live websites and internet archives that were identified as potential sources of dynamic relational information by querying ``Hurricane Katrina'' (and variants) in a popular internet search engine. Data collection included semi-automated downloading and saving of digital materials at roughly 12:00\textsc{am} until the source websites were exhausted each day from 08/24/05 and 09/05/05. Because many \texttt{url}s changed on daily basis, (i.e., through human error, archiving, or website reorganization) a human coder would verify each download and correct any missed attempts.

\section{Raw Material Preparation}
% This section is here because it was pulled from carter's original coding outline at the project's start

\subsection{Cataloging and Processing of the Information Sources}

This section details the general strategies employed to organize those documents that served as the raw source materials.  These documents are herein called \textit{source} documents, which refers to the \textit{source} organizations that produced them.

Each source organization utilized in this dataset was given a unique four digit identifier, in addition to both abbreviated and full names.  The first two digits of the ID identify state or organization kind.  State codes, which typically comprise the first two digits of the ID, are given in Tables~\ref{t.state.codes.1} and \ref{t.state.codes.2}.  Note that the District of Columbia is treated as a state here.

\begin{table}[H]
\begin{center}
\begin{tabular}{rll}
\toprule
\textbf{Code} & \textbf{State} & \textbf{Abbreviation} \\
\midrule
10 & Alabama & AL \\
11 & Alaska & AK \\
12 & Arizona & AZ \\
13 & Arkansas & AR \\
14 & California & CA \\
15 & Colorado & CO \\
16 & Connecticut & CT \\
17 & Delaware & DE \\
18 & District of Columbia & DC \\
19 & Florida & FL \\
20 & Georgia & GA \\
21 & Hawaii & HI \\
22 & Idaho & ID \\
23 & Illinois & IL \\
24 & Indiana & IN \\
25 & Iowa & IA \\
26 & Kansas & KS \\
27 & Kentucky & KY \\
28 & Louisiana & LA \\
29 & Maine & ME \\
30 & Maryland & MD \\
31 & Massachusetts & MA \\
32 & Michigan & MI \\
33 & Minnesota & MN \\
34 & Mississippi & MS \\
35 & Missouri & MO \\
\bottomrule
\end{tabular}
\caption{\label{t.state.codes.1}State Codes and Abbreviations}
\end{center}
\end{table}

\begin{table}[H]
\begin{center}
\begin{tabular}{rll}
\toprule
\textbf{Code} & \textbf{State} & \textbf{Abbreviation} \\
\midrule
36 & Montana & MT \\
37 & Nebraska & NE \\
38 & Nevada & NV \\
39 & New Hampshire & NH \\
40 & New Jersey & NJ \\
41 & New Mexico & NM \\
42 & New York & NY \\
43 & North Carolina & NC \\
44 & North Dakota & ND \\
45 & Ohio & OH \\
46 & Oklahoma & OK \\
47 & Oregon & OR \\
48 & Pennsylvania & PA \\
49 & Rhode Island & RI \\
50 & South Carolina & SC \\
51 & South Dakota & SD \\
52 & Tennessee & TN \\
53 & Texas & TX \\
54 & Utah & UT \\
55 & Vermont & VT \\
56 & Virginia & VI \\
57 & Washington & WA \\
58 & West Virginia & WV \\
59 & Wisconsin & WI \\
60 & Wyoming & WY\\
\bottomrule
\end{tabular}
\caption{\label{t.state.codes.2}State Codes and Abbreviations}
\end{center}
\end{table}

The second two digits of the ID variable generally categorize source organizations by their geographic scope of operations.  The ranges as defined below allow for multiple organizations to reside at each level when there exists multiple organizations that have as identical the first two digits in their ID:

\begin{table}[H]
\begin{center}
\begin{tabular}{rl}
\toprule
\textbf{Code Range} & \textbf{Description} \\
\midrule
00--09 & state level source\\
10--39 & city level source\\
40--69 & county level source\\
70--99 & other organization\\
\bottomrule
\end{tabular}
\end{center}
\end{table} 

If the source organization is a U.S. military organization, the first two digits of ID variable are `77' and the second two digits uniquely identify each organization.  If source is a non-profit organization, the first two digits of ID variable are `88' and the second two digits uniquely identify each organization.  If source is some other U.S. governmental agency or department, the first two digits of ID variable are `99' and the second two digits uniquely identify each organization.

The list of source organizations, along with their ID numbers and abbreviated names, is given in the  \texttt{katrina.bysrc} manual page within the R \texttt{katrina} data package.

\subsection{Cataloging of the Raw Materials}

This section describes the creation of ID numbers for each of the source documents (raw materials) from which the data arise.  ID numbers for these raw materials are eight digits in length.  The first four digits correspond to the source organization ID number, outlined previously. The next four digits were assigned sequentially by publication time, where possible in the format MMDD (there was no need to indicate the year, as all of these documents were published in 2005). This method of numbering was only feasible for sources that did not issue multiple documents in the same day. Otherwise the next four digits consist of `55' as the first two characters, and the last two are used to uniquely identify each document.

\subsection{Rendering Raw Materials to Text}

Source documents were typically in either \texttt{.doc}, \texttt{.pdf}, or \texttt{.html} format at the time of data collection. The following techniques were used to convert source documents into a common format (\texttt{.txt}):

\begin{itemize}
	\item \textbf{For \texttt{.pdf} files:} Two programs were used to implement the \texttt{.pdf} to \texttt{.txt} conversions: Adobe Acrobat 7.0 Professional and ABBYY FineReader 8.0 Professional Edition. Most conversions were implemented using Adobe Acrobat, but in cases where the conversion output was too difficult to read by a human, ABBYY FineReader was used instead to carry out the conversion. Many times Adobe Acrobat performed the nicer conversion between the two programs. Output was always specified as `Text (Accessible)' format in Adobe Acrobat. Both programs were used on a Microsoft Windows XP Professional machine.
	\item \textbf{For \texttt{.html} files:} All conversions from \texttt{.html} to \texttt{.txt} were performed using the Detagger 2.4.0.12 program for Microsoft Windows (\texttt{http://www.jafsoft.com/detagger/}). Detagger was configured to remove all html tags from the \texttt{.html} source file.
	\item \textbf{For \texttt{.doc} files:} The conversion of \texttt{.doc} files into \texttt{.txt} format was comparatively simple, requiring only that the user save the file as a \texttt{.txt} file from within Microsoft Word.
\end{itemize}

After conversion, all files were renamed with their corresponding `ID' variable value. For example, if a document's value on the `ID' variable is 99045503, its new name becomes \texttt{99045503.txt}.

\subsection{HTML Conversion of the Raw Materials}

Once all documents were converted into \texttt{.txt} format, copies of each file were made.  Into these copies were inserted basic html tags to render the files readable within a web browser, such as Mozilla Firefox (e.g., the <br></br> and <p></p> tags).  These copies were then given an \texttt{.html} extension.  These resulting HTML files served as the final versions of the documents which were processed and analyzed in order to produce this data package.

\section{Coding Methodology}

\subsection{Phase 1: Tagging of Organizational Actors}
A detailed explanation of the coding procedures used with the source documents is given in \citet{butts.acton.marcum:2010a}.

The preliminary identification of organizations included any formal organization of entities or a group of people who appeared to be acting as or serving as representatives of an organization.  Less obvious examples of organizations included (but were not limited to):
\begin{itemize}
\item Details, operations, missions, tasks, resources (e.g. a fencepost \& wire resource)
\item Teams, sections, task forces, branches, units, departments
\item Voluntary organizations/agencies
\item Personnel/representatives/liaisons of organizations
\item Schools, churches, farms, camps
\item Airports, air force bases, air reserve bases
\item Canteens
\item Stations
\item Emergency Operations Centers (EOCs) / Divisions/Departments of Emergency Management (DEMs) / Emergency Management Agencies (EMAs)
\item Emergency Support Functions (ESFs)
\item Courthouses
\item Facilities
\item General groups of people (inspectors, workers, crews, squads, patrols, troopers, troops)
\item Vehicles carrying out duties (rescue boats, trucks of ice)
\item Representatives of organizations (e.g., planning section chief, municipal contacts (if they're named specifically, not just a phone number))
\item Shops, retail locations, gas stations
\end{itemize}

This preliminary pass at identifying organizations yielded a high number of false positives (i.e., entities that initially appeared to be an organization, but upon further research, were not actually organizations at all).  Through the next step, named entity coding, the final set of actual organizations was realized.

\subsection{Phase 2: Coding of Named Entities as Organizational Actors}
This step finalized the set of organizations to be included in the data set.  The rule of thumb utilized here was that the organization in question must be clearly identified as an existing organization from other information sources (such as Internet searches or through identification in other documents), or must be a sub-unit of a larger organization.  Often times locating an organization's main website and/or address and contact information confirmed the organization's actual existence.  Some organizations, however, were difficult to verify.  Vague references to groups that initially appeared to be organizations but that could not be traced to any actual organizations in other information sources, were not included.  For example, `a team of electricians' would have been ignored if there was no clear reference to some parent organization or such from which the team arose.

There was some debate as to whether an Emergency Support Function (ESF) counted as an organization.  An ESF at face value is not an organization per se, rather it is a collection of organizations that are designated to manage a given task area during an emergency situation.  In the case of many of the SITREPs examined, an ESF was often referred to as engaging in activity with other organizations.  For example, ``ESF 13 deployed the Florida Army National Guard to Palm Beach this morning at 0900."  Because of this convention, an ESF was treated as an organization herein only when it was shown to be engaging in some specific relational activity with other organizations.

Another interesting aspect of these data is that many of the identified organizations are sub- or sub-sub-units of larger organizations.  Every effort was made to provide the most detailed information about an organization and its parental lineage.  For example, the state of Florida's Department of Health might have a coalition named ABC, which in turn sends a task force of specialists called XYZ to help Georgia with its recovery efforts.  Where the information is available, all parental lineages like this are recorded in these data.

There were times when it seemed fairly obvious to the coder that a given named entity truly was an organization, but the referent text left an element of ambiguity.  The named entity may even have appeared to be engaging in some relational activity with some other organization.  A problem with these SITREPs, largely an artifact of being converted from \texttt{.html} or \texttt{.pdf} to \texttt{.txt} and then back to \texttt{.html}, was that the formatting of tables and lists was sometimes lost.  This often led to scrambled or otherwise unreadable lists, which made deciphering a given named entity quite difficult.  In such cases when a named entity was difficult or impossible to properly code, the entity was not considered to be an organization.

It is also important to remember that these SITREPs were generated by humans, which means they are prone to error.  Because of this, a coder of documents like these must often make use of contextual clues in order to deal with quirky phrasing, abbreviations, misspellings, or even the improper naming of organizations.  While this introduces room for coding error, it is an essential part of the coding process.  Using contextual clues reduces error in the coding process to the extent that not employing them would result in more error (i.e., some prior information is better than none).

A clear example of this was evident in the Florida SITREPs.  Within several of these SITREPs, the abbreviation `DoF' was used to stand for `Division of Forestry'.  Then, in one instance, after several SITREPs prior had been using the `DoF' abbreviation, `DoF' was used to stand for `Division of Fire'.  This was initially quite troublesome, as it led the coder to question all previous coding of `DoF'.  Upon investigation, no such Division of Fire existed in Florida.  The Division of Fire fluke appeared in the same sections that the Division of Forestry references had appeared in SITREPs prior, and made no appearances anywhere else.  It was determined that this was a mistake made by the preparer of the SITREP, and the Division of Fire references were coded as references to the Division of Forestry.  This is only one example of many where some detective work and some context-based guessing revealed what was behind the confusion.  Of course, there are instances when the coder had no idea what the referent text was intended to mean.  This scenario points to the vital role that humans play, both in preparing the documents in the first place (which, as suggested here, are not immune to errors), as well as in interpreting ambiguities during the coding phase.

\subsection{Phases 3 and 4: Relationship Tagging and Coding}
In these data, two organizations are said to be relationally linked if the following actions were documented to have occurred between them within the time frame of the Hurricane response being reported in a given SITREP:
\begin{itemize}
\item There was an exchange of communication/ideas between organizations.  The most common example is the sharing/providing/requesting of information from one organization to another.
\item There was an exchange of manpower between organizations.  Common examples include the sending/receiving of assistance, liaisons, representatives of organizations to/from other organizations.
\item There was an exchange of material or financial support between organizations.
\item There was an exchange of power between organizations.  A common example is when one organization deploys/disengages or takes control of another organization.
\end{itemize}

It should be noted that the above detailed actions need not have been reciprocated between organizations.

For this pass through the raw materials in the coding procedure, no distinction was made about the kind or value of the relational ties between organizations.  All relations were coded as undirected, and self-reflexive loops (a case where an organization was documented to be engaging in relational activity with itself) were not allowed.

Once all instances of documented relational activity between the mentioned organizations were recorded, various adjacency matrices were generated to represent the relational data.

Matrices were compiled in three different ways: (1) as indicating the relational activity between organizations mentioned within the same  source document; (2) as indicating the relational activity between organizations mentioned by the same source organization; and (3) as indicating the relational activity between organizations which were documented as relating to one another within the same day (across all source organizations and source documents).  These three methods of computing matrices directly yielded the data contained in the \texttt{katrina.bydoc}, \texttt{katrina.bysrc}, and \texttt{katrina.bydate} data objects, respectively. A ``union rule'' version of network, representing all organizations and relations aggregated over the whole period, can be found in \texttt{katrina.combined}.

\subsection{Coding Organizational Lineage}
Where possible, for each organization identified in the response networks, we identified their parent organization lineages. Ties in these lineage networks were defined by ownership and/or administrative authority of one organization over another. The sources of the lineage data were varied and included: websites, promotional material, organization hierarchy charts, newspapers, and organizational informants cold-called by our coders. 

In many cases, parent organizations were easy to identify. Often it was the case that a given identified organization in the network was clearly a child of some broader organization or agency.  Within the federal government, for instance, it is not uncommon to find such a nested structure of departments, agencies, offices, divisions, and so forth. We smooth over the variability in how some municipalities are structured by assuming that, unless strictly defined, towns and cities are child organizations of counties and parishes, which are child organizations of states. In these data, we found that this set organizations could be traced to a maximum of order four parent organizations with a maximum lineage length of five organizations. The network object \texttt{katrina.lineage} contains the data for the parent organization lineage network.

Because not every parent organization was observed in the SITREPS from which we built the response networks, the \texttt{katrina.lineage} contains 260 more organizations than the master \texttt{katrina.combined} network. An vertex attribute called ``org.inset'' is included in the \texttt{katrina.lineage} that indicates whether or not an organization was part of the observed response networks.

Every effort was made to ensure that the lineage reflects the state of the world at the time of Hurricane Katrina.

\section{Organization Attributes}
Several key covariates were collected on each organization appearing in the SITREPS. Coders obtained this information through a combination of sources, including the SITREPS, organization websites, and published research articles. These variables are described in Table~\ref{t.covariates}, below.
\begin{sidewaystable}
	\begin{tabular}{lll}
		\textbf{Attribute name} & \textbf{Vector type} & \textbf{Description} \\
                first.appearance & numeric & The date of first appearance of each vertex.  Entry corresponds to location in \textbf{katrina.bydate}\\
		fema.region & numeric & The FEMA region codes for the headquarters location of each vertex\\
		group & numeric & The organization group code \\
		hq.city & character & The headquarters location (city only) for each vertex\\
		hq.state & character & The headquarters location (two letter U.S. state abbreviation only) for each vertex\\
		scale & numeric & The scale of operations code for each vertex\\
		source.org.status & numeric & Indicator for each vertex whether it is also a source organization.  If so, its unique source ID code.\\
		type & numeric & A vector containing the organization type code for each vertex.\\
                \hline
	\end{tabular}
\caption{\label{t.covariates}Table of Organization Covariates}
\end{sidewaystable}

We followed the coding schema of \citet{tierney2002llrg} for the scale and type variables. The ``group'' variable was created to mimic, as closely as feasible, the coding of \citet{lind&2008brdr} for their Katrina communication network.

%---Uncomment if we need to put the references on a separate page.
\clearpage
%\section{References} 
\bibliography{katrina.codebook.references}

\end{document}
